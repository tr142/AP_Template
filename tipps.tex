
\documentclass[twoside, 11pt, ngerman, a4paper, biblography=totoc]{scrartcl}

\usepackage{packages} 
%
%\lstset{%
%  language=TeX,                % choose the language of the code
%  basicstyle=\ttfamily\footnotesize\color{black} ,       % the size of the fonts that    are used for the code
%  tabsize=2,                      % sets default tabsize to 2 spaces
%  captionpos=b,                   % sets the caption-position to bottom
%  breaklines=true,                % sets automatic line breaking
%  breakatwhitespace=true,         % sets if automatic breaks should only happen at    whitespace
%  %keywordstyle=\color{white},
%  %commentstyle=\color{white},
%  %stringstyle=\color{white},
%  %numberstyle=\tiny\color{midgray},  % the style that is used for the line-numbers
%  stepnumber=2,                   % the step between two line-numbers. If it's 1, each line 
%                              % will be numbered
%  numbersep=5pt,                  % how far the line-numbers are from the code
%  frame=none,
%  %rulesepcolor=\color{darkgray},
%  lineskip={-1.5pt}, % single line spacing
%  aboveskip=1.5\bigskipamount,
%  belowskip=\smallskipamount,
%  escapechar={\@},
%  showstringspaces=false,
%  %backgroundcolor=\color{black}
%}

\begin{document}

\section{Beispiele und Tipps}
Hier ein paar Beispiele und Tipps für das erstellen eures Protokolls mit \LaTeX. Kein Anspruch auf Vollständigkeit.

\subsection{Allgemeines}
Overleaf und ChatGPT

\url{https://de.overleaf.com/learn/latex/How_to_Write_a_Thesis_in_LaTeX_(Part_1)\%3A_Basic_Structure}

\subsection{Abbildung}
Erstellt ihr Plots mit Python o.ä. lohnt es sich immer diese als Vektorgrafik in einer PDF abzuspeichern. Sorgt für kleien Datein und alles bleibt unverpixelt.\\

Um eine Abbildung einzubinden benötigen wir eine Figure Umgebung. Der Folgende Code
\begin{lstlisting}[language=Tex]
\begin{figure}[h]
    \centering
    \includegraphics[width = 0.8\textwidth]{../figures/example}
    \caption{Beispielabbildung. Bildunterschriften sollten die Abbildung beschreiben, so dass sie auch ohne Text verstandlich ist.}
    \label{fig:example}
\end{figure}
\end{lstlisting}
erzeugt diese Abbildung:
\begin{figure}[h]
    \centering
    \includegraphics[width = 0.8\textwidth]{../figures/example}
    \caption{Beispielabbildung. Bildunterschriften sollten die Abbildung beschreiben, so dass sie auch ohne Text verständlich ist.}
    \label{fig:example}
\end{figure}

Ein paar Details dazu:
\begin{itemize}
	\item \texttt{width = 0.8\textbackslash textwidth} bestimmt die Breite im Verhältnis zur Textbreite des Dokuments
	\item \texttt{../figures/example} ist der (relative) Pfad der Bilddatei. 
	\item \texttt{\textbackslash label\{fig:example\}} Gibt dem Objekt einen Namen, mit dem wir darauf zugreifen können, z.B. als Verweis: \cref{fig:example} (Verlinkung zum klicken)
	\item $[$h$]$ definiert die Postion der Abbildung im Dokument (h = here, also möglist nah am Text). Insgesamt ist es nicht notwendig, dass Abbildungen direkt an der Textstelle sind, v.a. wenn das zu halbleeren Seiten führt.
\end{itemize}

Subfigures ermöglichen das Neben/Übereinander stellen mehrerer Figures:
%
%\begin{lstlisting}[language=Tex]
%\begin{figure}
%     \centering
%     \begin{subfigure}[b]{0.3\textwidth}
%         \centering
%         \includegraphics[width=\textwidth]{example}
%         \caption{Subfigure 1}
%         \label{fig:subfig1}
%     \end{subfigure}
%     \hfill
%     \begin{subfigure}[b]{0.3\textwidth}
%         \centering
%         \includegraphics[width=\textwidth]{example}
%         \caption{Subfigure 2}
%         \label{fig:subfig2}
%     \end{subfigure}
%     \hfill
%     \begin{subfigure}[b]{0.3\textwidth}
%         \centering
%         \includegraphics[width=\textwidth]{example}
%         \caption{Subfigure 3}
%         \label{fig:subfig3}
%     \end{subfigure}
%        \caption{Three figures}
%        \label{fig:three graphs}
%\end{figure}
%\end{lstlisting}

%\begin{lstlisting}[language=Tex]
\begin{figure}[h!]
     \centering
     \begin{subfigure}[b]{0.3\textwidth}
         \centering
         \includegraphics[width=\textwidth]{example}
         \caption{Subfigure 1}
         \label{fig:subfig1}
     \end{subfigure}
     \hfill
     \begin{subfigure}[b]{0.3\textwidth}
         \centering
         \includegraphics[width=\textwidth]{example}
         \caption{Subfigure 2}
         \label{fig:subfig2}
     \end{subfigure}
     \hfill
     \begin{subfigure}[b]{0.3\textwidth}
         \centering
         \includegraphics[width=\textwidth]{example}
         \caption{Subfigure 3}
         \label{fig:subfig3}
     \end{subfigure}
        \caption{Three figures}
        \label{fig:three graphs}
\end{figure}
%\end{lstlisting}

\texttt{[t]\{0.3\textbackslash textwidth\}} definiert die Breite der Einzelnen Subfigures (alle: 0.3+0.3+0.3 = 0.9).

\subsection{Tabelle}

\subsection{Verweise und Zitate}
%\ref{}
%cref
%autoref

\subsection{Gleichungen und darstellung von Werten}

align, align* für Werte, equation,\\
Beispiel für eine Nummer für mehrere Zeilen\\
Physikalische Größen im Mathmode, Text normal
\begin{align}
	E_1&=A+B \label{eq:1}\\
	\begin{split}
		E_2&=(C-D)E_1 \label{eq:2}\\
		&\quad +[(1-R)+R(1-Y)\\
		&\quad +\pi(1-\delta)]E_2\\
		&\quad +F\cdot E_3
	\end{split}\\
	E_3 &=(\pi\cdot \chi)-(R\cdot E_1)-(RY\delta\cdot E_2) \label{eq:3}
\end{align}

\subsubsection{siunitx}
Das package \texttt{siunitx} 

Darstellung der Unsicherheiten als +/-, (), ...
\begin{verbatim}
	\SI{Bestwert(Unsicherheit)}{Einheit}
\end{verbatim}


Für optionen siehe z.B.: https://www.namsu.de/Extra/pakete/Siunitx.html

\subsection{Code}
Code can in LaTeX z.B. mit dem package \texttt{lstlisting} eingebunden werden. Entweder in der Umgebung
\begin{verbatim}
\begin{lstlisting}
		Put your code here.
\end{lstlisting}
\end{verbatim}
oder aus einer Datei 
\begin{verbatim}
		\lstinputlisting{filename.py}
\end{verbatim}
%\lstset
%https://en.wikibooks.org/wiki/LaTeX/Source_Code_Listings

\end{document}